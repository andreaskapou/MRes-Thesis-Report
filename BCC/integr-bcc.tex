\section{Bayesian Consensus Clustering} \label{integr-bcc-sect}
Motivated by the work of \citet{Lock2013}, this project is concerned with extending BCC to be applicable to NGS genomics data. Initial method was mainly intended for modelling \emph{continuous data}, and more specifically the observation model was assumed to follow a Gaussian distribution. Hence, the approach was mainly demonstrated on TCGA datasets, which were generated from \emph{microarray hybridization} experiments \citep{Babu2004}. However, with the advent of high-throughput sequencing methods, the data follow different probability distributions since most experiments return \emph{count data} (see \emph{Chapter \ref{dataset-chapter}}). Thus, we need to extend BC, so it can model data that follow different probability distributions, such as \emph{Binomial} and \emph{Poisson}.

For the explanation of the BCC model, the notation will be the same as the one used in \citet{Lock2013}. 

To perform integrative clustering for M different data sources $\mathbb{X}_{1},..., \mathbb{X}_{M}$, the Finite Dirichlet Mixture Model (FDMM) (see \emph{Section \ref{back-fdmm-s}}) is extended, which leads us to the Bayesian Consensus Clustering model. BCC assumes that there are N common objects for each data source, where $X_{mn}$ denotes the data for object $n$ from source $m$. The data sources $\mathbb{X}_{m}$ can have any disparate structure, and each of them requires an arbitrary observation model $f_{m}(X_{mn}|\theta_{m})$, where $\theta_{m}$ denotes the model parameters for data source $m$.

Let $\mathbb{L}_{m} = (L_{m1},...,L_{mN})$ denote the source-specific clusterings and $\mathbb{C} = (C_{1},...,C_{N})$ denote the overall clustering of the data. The model assumes that both $\mathbb{L}_{m}$ and $\mathbb{C}$ will have the same number of total clusters K, thus $L_{mn} \in \lbrace 1,...,K \rbrace$ and $C_{n} \in \lbrace 1,...,K \rbrace$, will denote the source-specific and the overall mixture components for observation $X_{mn}$, respectively.

The source-specific and the overall clusterings are related to each other, and the strength of the relation is given by a dependence function $v(L_{mn}, C_{n}, \alpha_{m})$. The \emph{dependence function} $v$ has the following form:
\begin{equation}
	v(L_{mn}, C_{n}, \alpha_{m}) = \left\{
	\begin{array}{l l}
		\alpha_{m},\quad \quad \quad if\quad C_{n} = L_{mn}\\
		\frac{1-\alpha_{m}}{K-1},\quad \quad otherwise
	\end{array}\right.
\end{equation}
where $\alpha_{m} \in [\frac{1}{K}, 1]$ is the called the \emph{adherence} parameter and controls the adherence of data source $\mathbb{X}_m$ to the overall clustering $\mathbb{C}$. Intuitively, it explains how much does the source specific clustering for source $m$ agrees with the overall clustering $\mathbb{C}$. For example, if $\alpha_{m} = 1$, then $\mathbb{L}_{m} = \mathbb{C}$, thus we have a perfect agreement between the source-specific and the overall clustering assignments. On the other extreme, if $\alpha_{m} = \frac{1}{K}$, then there is no relationship between $\mathbb{L}_{m}$ and $\mathbb{C}$. The adherence parameters $\alpha_{m}$ are estimated from the data, and they can either be equal for all data sources, \ie $\alpha_{1} = ... = \alpha_{M}$, or unique, and thus a different $\alpha_{m}$ for each data source $m$ needs to be estimated. 


The graphical representation for the BCC model is shown below, where $\mathcal{G}_{0}$ denotes the prior probability distribution for the parameters $\theta_{mk}$, and $\mathcal{TB}(\cdot, \cdot, \frac{1}{K})$ denotes the \emph{Beta} distribution truncated below by $\frac{1}{K}$.

\vspace*{5mm}
\begin{minipage}{0.5\textwidth}%
  \hfill
  \begin{center}
	% model_pca.tex
%
% Copyright (C) 2012 Jaakko Luttinen
%
% This file may be distributed and/or modified
%
% 1. under the LaTeX Project Public License and/or
% 2. under the GNU General Public License.
%
% See the files LICENSE_LPPL and LICENSE_GPL for more details.

% PCA model

%\beginpgfgraphicnamed{model-pca}
\begin{tikzpicture}

  % Define nodes
  \node[obs]                              (X) {$X_{mn}$};
  \node[latent, above=of X] 		(L) {$L_{mn}$};
  \node[latent, above=of L] 		(a) {$\alpha$};
  \node[latent, right=1.4cm of L]  (C) {$C_{n}$};
  \node[latent, above=of C] 		(p) {$\pi$};
  \node[latent, right=1cm of C]  (t) {$\theta_{mk}$};
  \node[latent, above=of t] 		(G) {$\mathcal{G}_{0}$};

  % Connect the nodes
  \edge {L,t} {X} ; %
  \edge {a,C} {L} ; %
  \edge {p} {C} ; %
  \edge {G} {t} ; %

  % Plates
  \plate {} {(t)} {$K$};
  \plate {yx} {(L)(X)} {$M$};
  \plate {} {(C)(X)(yx.north west)(yx.south west)} {$N$};

\end{tikzpicture}
%\endpgfgraphicnamed

%%% Local Variables: 
%%% mode: tex-pdf
%%% TeX-master: "example"
%%% End: 

	\emph{Graphical Model of BCC.}
  \end{center}
\end{minipage}
%\hfill
\begin{minipage}{0.4\textwidth}%\raggedright
  \begin{equation*}
  	\begin{aligned}
  		\mathbf{\pi} \; & \sim \; \mathcal{D}ir(\delta) \\
  		\alpha \; & \sim \; \mathcal{TB}(\mathit{a}, \beta, \frac{1}{K}) \\
  		\theta_{mk} \; & \sim \mathcal{G}_{0}(\cdot | \mathcal{H}) \\
  		C_{n} \mid \mathbf{\pi} \; & \sim \; \mathcal{C}at(\mathbf{\pi}) \\
  		L_{mn} \mid \alpha_{m}, C_{n} \; & \sim \; v(L_{mn}, C_{n}, \alpha_{m}) \\
  		X_{mn} \mid L_{mn}=k,\theta_{mk} \; & \sim \; f_{m}(\cdot | \theta_{mk}) 
  	\end{aligned} 
  \end{equation*} 
\end{minipage}
\vspace*{5mm}

\subsection{Bayesian inference for BCC model}\label{integr-bayes-inference-subsect}
The full joint distribution of the BCC model, by looking at the graphical representation factorizes as follows:
\begin{equation}%\scriptstyle
  \begin{aligned}
	P(\mathbb{X}, \mathbb{L}, \mathbb{C}, \pi , \Theta , \alpha) & = P(\mathbb{X}|\mathbb{L},\Theta) P(\mathbb{L}|\mathbb{C},\alpha) P(\mathbb{C}|\pi) P(\alpha) P(\pi) P(\Theta) \\
	  & = \bigg[\prod_{n}\bigg(\prod_{m} P(X_{mn}|L_{mn},\theta_{m,L_{mn}}) P(L_{mn}|C_{n},\alpha_{m})\bigg) P(C_{n}|\pi)\bigg] \\
	  & \; \quad \quad \quad \quad \quad \quad \quad \quad \quad \quad \quad \quad P(\alpha) P(\pi) \bigg(\prod_{m}\prod_{k} P(\theta_{mk})\bigg)
  \end{aligned}
\end{equation}
where the prior distribution $\mathcal{G}_{0}$ and the hyper-parameters ($\delta, \mathit{a}, \beta, \mathcal{H}$) are omitted for clarity.

To perform inference in the Bayesian framework, the posterior distribution of the variables needs to be estimated. By applying Bayes' theorem and conditioning on the observed data $\mathbb{X}$ from all data sources, the posterior distribution is the following:

\begin{equation}%\scriptstyle
	\begin{aligned}
	P(\mathbb{L},\mathbb{C},\pi,\Theta,\alpha | \mathbb{X}) & = \frac{P(\mathbb{X}|\mathbb{L},\mathbb{C},\pi,\Theta,\alpha) P(\mathbb{L},\mathbb{C},\pi,\Theta,\alpha)}{p(\mathbb{X})} \\
	& = \frac{P(\mathbb{X}|\mathbb{L},\Theta) P(\mathbb{L}|\mathbb{C},\alpha) P(\mathbb{C}|\pi) P(\alpha) P(\pi) P(\Theta)}{p(\mathbb{X})}
	\end{aligned}
\end{equation}
where we use the fact that $\mathbb{X}$ is conditionally independent of $\mathbb{C}, \pi$ and $\alpha$ given $\mathbb{L}$, \ie $\mathbb{X} \bigCI \mathbb{C}, \pi, \alpha \mid \mathbb{L}$, by looking at the \emph{Markov blanket} of $\mathbb{X}$. 

This posterior is intractable to compute, thus we need to resort to an approximation scheme. The algorithm that will be used is \emph{Gibbs sampling} \citep{Geman1984}. Deriving Gibbs sampling for this model requires deriving an expression for the \emph{full conditional} distribution of every random variable. From the graphical representation of the BCC model, we can infer the conditional independencies between the latent variables by looking at their \emph{Markov blanket}. 

For mathematical convenience, \emph{conjugate priors} should be used over the parameters, which would allow us to write analytically the full conditional distributions, and also draw samples from them. Thus:
\begin{itemize}
	\item $\pi_{k} \sim \mathcal{D}ir(\mathbf{\delta})$. For the mixing proportions a natural choice is a Dirichlet prior distribution, and the hyper-parameter vector $\mathbf{\delta}$ is set to $(1,...,1)$ so the prior is uniformly distributed in the $K-1$ simplex.
	\item $\alpha_{m} \sim \mathcal{TB}(\mathit{a}, \beta, \frac{1}{K})$. Which means that $\alpha_{m}$ follows a $Beta$ distribution with parameters $\mathit{a}, \beta$ truncated below by $\frac{1}{K}$.
	\item $\theta_{mk} \sim \mathcal{G}_{0}$. Where $\mathcal{G}_{0}$ is a conjugate prior distribution, so that sampling from the full conditional distribution $\mathcal{G}_{0}(\theta_{mk}|\mathbb{X}_{m}, \mathbb{L}_{m})$ is feasible.
\end{itemize}

\noindent\textbf{Gibbs sampling algorithm for BCC}: 
\begin{itemize}
\item {Initially, at simulation step 0, parameter values are chosen arbitrarily.}
\item {Suppose at simulation step $\tau$ we have sampled values $\mathbb{L}^{(\tau)}, \Theta^{(\tau)}, \alpha^{(\tau)}, \mathbb{C}^{(\tau)}, \pi^{(\tau)}$.}
\item { At simulation step $\tau + 1$, the Gibbs sampler continues as follows:
\begin{equation*}%\scriptstyle
  \begin{aligned}
    L_{mn}^{(\tau+1)} \mid X_{mn}, \Theta_{m}^{(\tau)}, C_{n}^{(\tau)}, \alpha_{m}^{(\tau)} & \sim v(L_{mn}, C_{n}, \alpha_{m}) f_{m}(X_{mn}|\theta_{mk}) \\
	\theta_{mk}^{(\tau+1)} \mid \mathbb{L}_{m}^{(\tau+1)}, \mathbb{X}_{m} & \sim \mathcal{G}_{0}(\theta_{mk} \mid \mathbb{X}_{m}, \mathbb{L}_{m} = k; \mathcal{H}) \\
%	\alpha_{m}^{(\tau+1)} \mid \mathbb{L}_{m}^{(\tau+1)}, \mathbb{C}^{(\tau)} & \sim \mathcal{TB} \big(\mathit{a}+\sum_{n}\mathbbm{1}(L_{mn}=k,C_{n}=k), \\
%	  & \quad \quad \quad \; \beta + N -\sum_{n}\mathbbm{1}(L_{mn}=k,C_{n}=k), \frac{1}{K}\big) \\
	\alpha_{m}^{(\tau+1)} \mid \mathbb{L}_{m}^{(\tau+1)}, \mathbb{C}^{(\tau)} & \sim \mathcal{TB} \big(\mathit{a} + Q_{mk}, \beta + N - Q_{mk}, \frac{1}{K}\big) \\
	C_{n}^{(\tau+1)} \mid L_{mn}^{(\tau+1)}, \alpha^{(\tau+1)}, \pi^{(\tau)} & \sim \pi_{k} \prod_{m}v(L_{mn}, C_{n}, \alpha_{m}) \\
	\pi_{k}^{(\tau+1)} \mid \mathbb{C}^{(\tau+1)} & \sim \mathcal{D}ir\big(\delta + \sum_{n}\mathbbm{1}(C_{n}=k)\big)
  \end{aligned}
\end{equation*}
where $Q_{mk} = \sum_{n}\mathbbm{1}(L_{mn}=k,C_{n}=k)$, and $\mathbbm{1}(\cdot,\cdot)$ is the indicator function, equal to 1 if the quantities inside the function are satisfied, and 0 otherwise.
}
\end{itemize}

\subsection{Extending BCC model for NGS data}\label{integr-extension-subsect}
As it was aforementioned, with the advent of high-throughput sequencing methods, BCC needs to be extended so it can model data that follow different probability distributions.

Due to conditional independence structure of the mixture models, changing the observation model for data source $\mathbb{X}_{m}$ does not affect the mixing proportions $\pi$, the adherence parameter $\alpha$, and the overall clustering assignments $\mathbb{C}$. Thus, modelling and subsequently inferring these variables remains the same as it was explained in the previous section.

The source-specific clustering assignments $\mathbb{L}_{m}$ depend on the observation model of the data sources $\mathbb{X}_{m}$. Hopefully, the update equations for $\mathbb{L}_{m}$ on each Gibbs simulation step are straightforward to adapt, since we just need to evaluate $f_{m}(X_{mn}|\theta_{m})$ pointwise, which for most known probability distribution is easy to compute.

Thus, the focus of this section is mainly in modelling, \ie defining a prior distribution, and inferring the parameters $\theta_{m}$ of the observation model. 

\subsubsection*{Gaussian observation model}
For completeness we provide details for the Gaussian observation model, as it was described in \citet{Lock2013}. When the data source $\mathbb{X}_{m}$ follows a Gaussian mixture distribution, we have:
\begin{equation}
	X_{mn} \mid L_{mn} = k, \theta_{mk} \sim \mathcal{N}\big(\mu_{mk}, \tau_{mk}^{-1}\big)
\end{equation}
where $\mathcal{N}$ denotes the \emph{Normal} or \emph{Gaussian} distribution, $\mu_{mk}$ is the \emph{mean} parameter, and $\tau_{mk}$ is the \emph{precision} parameter, \ie the reciprocal of the variance, $\tau_{mk} = \frac{1}{\sigma_{mk}^{2}}$.

We choose a \emph{Normal-Gamma} conjugate prior distribution for the parameters $\theta_{mk} = (\mu_{mk}, \tau_{mk})$, that is:
\begin{equation}
	\theta_{mk} \sim \mathcal{NG}amma\big(\mu_{m0}, \tau_{m0}, \mathit{a}_{m0}, \beta_{m0}\big)
\end{equation}
where hyper-parameters $\mu_{m0}, \tau_{m0}$ are the same as defined above, and $\mathit{a}_{m0}, \beta_{m0}$ are the \emph{shape} and \emph{rate} hyper-parameters of the \emph{Gamma} distribution, respectively.

Thus, on each iteration step of Gibbs algorithm, we have the following updates for each parameter:
\begin{equation}
  \begin{aligned}
  	\tau_{mk} \mid \mathbb{L}_{m}=k, \mathbb{X}_{m} \;& \sim \;\mathcal{G}amma\big(\hat{\mathit{a}}_{m0}, \hat{\beta}_{m0}\big) \\
	\mu_{mk} \mid \tau_{mk}, \mathbb{L}_{m}=k, \mathbb{X}_{m} \; & \sim \; \mathcal{N}\big(\hat{\mu}_{m0}, (\hat{\tau}_{m0} \tau_{mk})^{-1}\big)
  \end{aligned}
\end{equation}
The posterior parameter values for the $\mathcal{NG}(\cdot,\cdot,\cdot,\cdot)$ distribution are calculated as follows:
\begin{equation}
  \begin{aligned}
  	\hat{\mu}_{m0} \; &= \; \frac{\tau_{m0}\mu_{m0} + N_{mk}\mathcal{X}_{mk}}{\tau_{m0} + N_{mk}}\\
  	\hat{\tau}_{m0} \; &= \; \tau_{m0} + N_{mk}\\
  	\hat{\mathit{a}}_{m0} \; &= \; \mathit{a}_{m0} + \frac{N_{mk}}{2}\\
  	\hat{\beta}_{m0} \; &= \; \beta_{m0} + \frac{SSD_{mk}}{2} + \frac{N_{mk}\tau_{m0}(\mathcal{X}_{mk} - \mu_{m0})}{2(N_{mk}+\tau_{m0})}
  \end{aligned}
\end{equation}
where: 
\begin{equation}
  \begin{aligned}
		N_{mk} \; &= \; \sum_{n}\mathbbm{1}(L_{mn}=k)\\ 
		\mathcal{X}_{mk} \; &= \; \frac{\sum_{n}\mathbbm{1}(L_{mn}=k)X_{mn}}{N_{mk}} \\
		SSD_{mk} \; &= \; \sum_{n}\mathbbm{1}(L_{mn}=k)(X_{mn}-\mathcal{X}_{mk})^{2}
  \end{aligned}
\end{equation} 

If the data source $\mathbb{X}_{m}$ is $D_{m}$-dimensional, then we can use either a $D_{m}$ dimensional \emph{Normal-Gamma} prior, if we assume a $D_{m} \times D_{m}$ \emph{diagonal precision} matrix, or a \emph{Normal-Wishart} prior, if we assume a $D_{m} \times D_{m}$ \emph{full precision} matrix \cite[Ch. 2]{Bishop2006}.

\subsubsection*{Binomial observation model}
When the data source $\mathbb{X}_{m}$ follows a Binomial mixture distribution, we have:
\begin{equation}
	X_{mn} \mid L_{mn} = k, \theta_{mk} \sim \mathcal{B}inom\big(t_{mk}, \rho_{mk}\big)
\end{equation}
where $t_{mk}$ is \emph{known} and denotes the total number of experiments, and $\rho_{mk}$ is the parameter for the probability of success.

We choose a \emph{Beta} conjugate prior distribution for the parameter $\rho_{mk}$, that is:
\begin{equation}
	\rho_{mk} \sim \mathcal{B}eta\big(\mathit{a}_{m0}, \beta_{m0}\big)
\end{equation}
where $\mathit{a}_{m0}$ and $\beta_{m0}$ are the \emph{shape} hyper-parameters of the \emph{Beta} distribution.

Thus, on each iteration step of Gibbs algorithm, we have the following update for the $\rho_{mk}$ parameter:
\begin{equation}
  \begin{aligned}
  	\rho_{mk} \mid \mathbb{L}_{m}=k, \mathbb{X}_{m} \;& \sim \;\mathcal{B}eta\big(\hat{\mathit{a}}_{m0}, \hat{\beta}_{m0}\big) \\
  \end{aligned}
\end{equation}
where the posterior parameter values for the $\mathcal{B}eta(\cdot,\cdot)$ distribution are calculated as follows:
\begin{equation}
  \begin{aligned}
  	\hat{\mathit{a}}_{m0} \; &= \; \mathit{a}_{m0} + \sum_{n}\mathbbm{1}(L_{mn}=k)X_{mn} \\
  	\hat{\beta}_{m0} \; &= \; \beta_{m0} +  \sum_{n}\mathbbm{1}(L_{mn}=k)(r_{mn} - X_{mn})
  \end{aligned}
\end{equation}

\subsubsection*{Poisson observation model}
When the data source $\mathbb{X}_{m}$ follows a Poisson mixture distribution, we have:
\begin{equation}
	X_{mn} \mid L_{mn} = k, \theta_{mk} \sim \mathcal{P}ois\big(\lambda_{mk}\big)
\end{equation}
where $\lambda_{mk}$ denotes the mean and variance parameter.

We choose a \emph{Gamma} conjugate prior distribution for the parameter $\lambda_{mk}$, that is:
\begin{equation}
	\lambda_{mk} \sim \mathcal{G}amma\big(\mathit{a}_{m0}, \beta_{m0}\big)
\end{equation}
where $\mathit{a}_{m0}$ and $\beta_{m0}$ are the \emph{shape} and \emph{rate} hyper-parameters of the \emph{Gamma} distribution.

Thus, on each iteration step of Gibbs algorithm, we have the following update for the $\lambda_{mk}$ parameter:
\begin{equation}
  \begin{aligned}
  	\lambda_{mk} \mid \mathbb{L}_{m}=k, \mathbb{X}_{m} \;& \sim \;\mathcal{G}amma\big(\hat{\mathit{a}}_{m0}, \hat{\beta}_{m0}\big) \\
  \end{aligned}
\end{equation}
where the posterior parameter values for the $\mathcal{G}amma(\cdot,\cdot)$ distribution are calculated as follows:
\begin{equation}
  \begin{aligned}
  	\hat{\mathit{a}}_{m0} \; &= \; \mathit{a}_{m0} + \sum_{n}\mathbbm{1}(L_{mn}=k)X_{mn} \\
  	\hat{\beta}_{m0} \; &= \; \beta_{m0} +  \sum_{n}\mathbbm{1}(L_{mn}=k)
  \end{aligned}
\end{equation}
