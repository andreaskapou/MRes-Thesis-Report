\section{Finite Dirichlet Mixture Models} \label{fdmm-s}

\begin{minipage}{0.6\textwidth}%
  \hfill
  \begin{center}
	% model_pca.tex
%
% Copyright (C) 2012 Jaakko Luttinen
%
% This file may be distributed and/or modified
%
% 1. under the LaTeX Project Public License and/or
% 2. under the GNU General Public License.
%
% See the files LICENSE_LPPL and LICENSE_GPL for more details.

% PCA model

%\beginpgfgraphicnamed{model-pca}
\begin{tikzpicture}

  % Define nodes
  \node[obs]                              (X) {$X_{n}$};
  \node[latent, above=of X] 		(C) {$C_{n}$};
  \node[latent, above=of C]  	(p) {$\pi$};
  \node[latent, right=1cm of C]  (t) {$\theta_{k}$};
  \node[latent, above=of t] 		(G) {$G_{0}$};

  % Connect the nodes
  \edge {C,t} {X} ; %
  \edge {p} {C} ; %
  \edge {G} {t} ; %

  % Plates
  \plate {} {(t)} {$K$};
  \plate {} {(C)(X)} {$N$};

\end{tikzpicture}
%\endpgfgraphicnamed

%%% Local Variables: 
%%% mode: tex-pdf
%%% TeX-master: "example"
%%% End: 

	\emph{Graphical Model of FDMM.}
  \end{center}
\end{minipage}
%\hfill
\begin{minipage}{0.1\textwidth}%\raggedright
  \begin{equation*}
  	\begin{aligned}
  		\mathbf{\pi} \; & \sim \; Dir(\mathbf{\delta}) \\
  		z_{i}|\mathbf{\pi} \; & \sim \; Cat(\mathbf{\pi}) \\
  		\theta_{k} \; & \sim \mathcal{G}_{0}; \mathcal{H} \\
  		x_{i}|z_{i}=k,\theta_{k} \; & \sim \; p(\cdot | \theta_{k})  
  	\end{aligned} 
  \end{equation*} 
\end{minipage}

\vspace*{5mm}
The full joint distribution of the FDMM model by looking at the graphical representation factorizes as follows:
\begin{equation}%\scriptstyle
	p(\mathbf{X},\mathbf{Z},\mathbf{\pi},\Theta;\delta,\mathcal{H}) = p(\mathbf{X}|\mathbf{Z},\Theta) p(\mathbf{Z}|\pi) p(\pi|\delta) p(\Theta |\mathcal{G}_{0}; \mathcal{H})
\end{equation}

where $\mathcal{H}$ is the set of all the hyper-parameters of the $\mathcal{G}_{0}$ prior distribution related to the parameters $\Theta$, and $\mathbf{\delta}$ is a K-dimensional vector with the hyper-parameters of the \emph{Dirichlet} prior. 

To do inference in the Bayesian framework, the posterior distribution of the parameters needs to be computed. By applying the Bayes Rule and conditioning on the observed data $\mathbf{X}$, the posterior distribution is simply proportional to the full joint. Thus:
 
\begin{equation}%\scriptstyle
  \begin{aligned}
	p(\mathbf{Z},\mathbf{\pi},\Theta|\mathbf{X} ;\delta,\mathcal{H}) & = \frac{p(\mathbf{X}|\mathbf{Z},\mathbf{\pi},\Theta) p(\mathbf{Z},\mathbf{\pi},\Theta ;\delta,\mathcal{H})}{p(\mathbf{X})} \\
	   & \propto p(\mathbf{X}|\mathbf{Z},\mathbf{\pi},\Theta) p(\mathbf{Z},\mathbf{\pi},\Theta ;\delta,\mathcal{H}) \\
	   & = p(\mathbf{X}|\mathbf{Z},\Theta) p(\mathbf{Z}|\pi) p(\pi|\delta) p(\Theta |\mathcal{G}_{0}; \mathcal{H}) \\
	   & = \bigg(\prod\limits_{i=1}^{N} p(x_{i}|z_{i},\theta_{z_{i}}) p(z_{i}|\pi)\bigg) p(\pi|\delta) \bigg(\prod\limits_{k=1}^{K} p(\theta_{k} |\mathcal{G}_{0}; \mathcal{H})\bigg)
  \end{aligned}
\end{equation}
where we use the fact that $\mathbf{X}$ is conditionally independent of $\pi$ given $\mathbf{Z}$, \ie $\mathbf{X} \bigCI \pi \;| \; \mathbf{Z} $. 