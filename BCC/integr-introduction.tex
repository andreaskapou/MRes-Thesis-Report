\section{Introduction} \label{integr-intro-sect}
Modern high-throughput genomics platforms generate large amounts of biomedical data from different sources, and these data are used to measure diverse, but often related and complementary, information. Even though these data are becoming widely available, \eg ENCODE datasets \citep{Dunham2012}, integrated computational analysis is still a challenging task and crucial in order to interpret and uncover biological regulatory mechanisms \citep{Park2009}.   

The main aim of this chapter, is to propose a statistical method that integrates the heterogeneous types of high-throughput biological data using an unsupervised integrative data modelling approach. The standard approaches in integrative clustering either perform a \emph{separate clustering} followed by a post hoc integration \citep{Wang2011} or incorporate all data sources simultaneously and generate a single \emph{'joint' clustering} \citep{Kormaksson2012, Mo2013}. The problem with the separate clustering approach is that it may lack power and will not be able to capture inter-source associations between the data. On the other hand 'joint' clustering ignores the heterogeneity of the data sources and may not capture source-specific features. Thus, flexible clustering methods need to be developed that can simultaneously integrate information from different data and can also capture the underlying structural similarities across the data sources.