\section{Introduction} \label{data-intro-sect}
This chapter is concerned with introducing data produced from Next Generation Sequencing (NGS) technology, also termed as high-throughput sequencing. \emph{DNA sequencing} is the process of determining the complete order of DNA nucleotides of an organism's genome at a single time. Until recently, the method of choice for DNA sequencing was the \emph{chain termination} method developed by \citet{Sanger1977}. This technology, and its variants, even though they are widely adopted, they have inherent limitations in throughput, scalability, speed and cost. 

NGS technology \citep{Shendure2008, Mardis2008}, performs massively parallel sequencing on DNA fragments, producing thousands or millions of DNA fragment sequences, called \emph{reads}, concurrently. This yields substantially higher throughput and allows for entire genomes to be sequenced within a day. By reducing the cost by over two orders of magnitude from traditional Sanger sequencing methods \citep{Shendure2008}, DNA sequencing became accessible for smaller labs, allowing for rapid development of applications in fields related to biological and biomedical sciences. Despite the widespread take up of the NGS technology, computational data analysis is still a challenging task.