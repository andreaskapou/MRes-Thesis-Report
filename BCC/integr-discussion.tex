\section{Discussion} \label{integr-discussion-sect}
In this chapter we considered the problem of performing integrative clustering on multisource data, and more specifically on data that are produced from NGS technology.

BCC model, proposed by \citet{Lock2013}, is a flexible clustering method that simultaneously integrates information from different data and also captures the underlying structural similarities across data sources. As it was shown, BCC can be thought as a flexible bridge between the extremes of separate and joint clusterings, since it introduces a parameter for controlling the adherence of each data source to the overall clustering. Also, by performing an integrative statistical modelling of both the source-specific and overall clusterings simultaneously, makes the method more powerful and allows for uncertainty in all parameters.

The main contribution of this project was to extend the BCC model to be applicable to NGS genomics data. A \emph{Binomial} observation model was introduced for clustering DNA methylation data generated from RRBS experiments, and a \emph{Poisson} observation model for clustering gene expression data generated from RNA-Seq experiments. 

The analysis that we conducted on the synthetic data, confirm the the statistical power of the method on estimating the parameters with high accuracy. 



%It assumes that there is a simple dependence between the data sources BCC model is useful in problems where we seek an overall clustering between data sources assumes that there is a 
%By performing source-specific and overall clustering simultaneously in the same model in a statistical way, permits 
%The statistical modelling of both the source-specific and overall clusterings in the same framework allows to get uncertainty in all parameters,   
