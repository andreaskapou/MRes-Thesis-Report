\section{Modelling DNA methylation profiles} \label{model-meth-profiles-s}
DNA methylation data generated from NGS technology can be modelled with a Binomial distribution. Assume that \emph{t} is the total number of reads that are mapped to a specific CpG site, and that the 5-Methylcytosine were found in \emph{m} of these reads. Then for each CpG site we have the read proportion \emph{(m,t)}, and we assume that:
\begin{equation} \label{binom-1d-f-meth}
	m \sim Binom(t, p)
\end{equation}
where \emph{p} is the unknown methylation level of the CpG site.

We are interested in modelling methylation profiles across different genomic regions, \eg promoter regions, hence the data can be represented as a set of vectors, whose dimensionality \emph{L} depends on the number of the individual CpG sites for that specific region. Thus, each region \emph{i} can be represented by a vector $\mathbf{y}_{i}$ of dimensionality $L_{i}$, and each entry of the vector consists of the tuple:
\begin{equation}
	%n_{i}^{l} = \frac{m_{i}^{l}}{t_{i}^{l}}
	y_{il} = (m_{il},t_{il})
\end{equation}
where, $m_{il}$ is the number of 5-Methylcytosine reads on $l$-th CpG site in region $i$, and $t_{il}$ is the total number of reads on $l$-th CpG site in region $i$. 

Hence, we can formulate our problem as a \emph{regression} problem, where we try to fit a function $\mathbf{f}$, of some specific form, to the observations $\mathbf{y}$. A promising approach for modelling the methylation profile for a given region is to use a \emph{Gaussian Process} for classification \citep{Rasmussen2006}. Even though this approach would be ideal, its time complexity of $O(n^{3})$ is prohibitive for our purposes. Thus, we have to resort to modelling the methylation profiles using $n^{th}$ degree polynomial functions.

More specifically, let $\mathbf{x}_{i}$ be the genomic region of interest, $\mathbf{y}_{i}$ be the vector of observations in this region (\ie proportions of methylated reads to the total reads at each site), and let $f(\mathbf{x}_{i})$ be a latent function representing the methylation profile at that specific genomic region. Since the observed methylation data are actually the fraction of total reads that contained methylated cytosines, each entry in the region will take values in the $[0, 1]$ interval, thus, we introduce a latent function $g(\mathbf{x}_{i})$, \eg $g(\mathbf{x}_{i}) = \alpha \mathbf{x}_{i}^{2} + \beta \mathbf{x}_{i} + c$, defined as the \emph{inverse probit transformation} of $f(\mathbf{x}_{i})$. In other words, $f(\mathbf{x}_{i})$ is the probit transformation of $g(\mathbf{x}_{i})$:
\begin{equation} \label{probit-transform-f-meth}
	f(\mathbf{x}_{i}) = \Phi(g(\mathbf{x}_{i}))
\end{equation}
where $\Phi(x)$ is the Cumulative Distribution Function (CDF) of the standard normal distribution:
\begin{equation} \label{cdf-stand-normal-f-meth}
	\Phi(x) = \frac{1}{\sqrt{2\pi}} \int_{-\infty}^{x} e^{-t^{2}/2}dt
\end{equation}

Let $\mathbf{f}_{i} = f(\mathbf{x}_{i})$ and $\mathbf{g}_{i} = g(\mathbf{x}_{i})$ be shorthand for the values of the latent functions.

Given the values of the latent function $\mathbf{f}_{i}$, the observations at each CpG site for that region are independent Binomial variables, as shown in \emph{Eq. \ref{binom-1d-f-meth}}, so the joint likelihood factorizes and we have:
\begin{equation} \label{likel-binom-prob-f-meth}
  \begin{split}
	p(\mathbf{y}_{i}|\mathbf{f}_{i}) & = \prod_{l=1}^{L} p(y_{il}|f_{il}) \\
							 & = \prod_{l=1}^{L} Binom(t_{il}, f_{il}) \\
							 & = \prod_{l=1}^{L} Binom\big(t_{il}, \Phi(g_{il})\big) \\
							 & = \prod_{l=1}^{L} \binom{t_{il}}{m_{il}} \Phi(g_{il})^{m_{il}} (1 - \Phi(g_{il})\big)^{t_{il} - m_{il}}
  \end{split}
\end{equation}

From its final form, we refer to this function as the \emph{Binomial distributed Probit regression function}. In practice, the likelihood is computed in the log space, due to numerical issues when multiplying many probabilities of small numbers leading to underflow errors. Thus, the joint log likelihood for region \emph{i} is:
\begin{equation} \label{likel-binom-prob-log-f-meth}
  \begin{split}
	\log p(\mathbf{y}_{i}|\mathbf{f}_{i}) & = \sum_{l=1}^{L} \log p(y_{il}|f_{il}) \\
				& = \sum_{l=1}^{L} \log \bigg(\binom{t_{il}}{m_{il}} \Phi(g_{il})^{m_{il}} \big(1 - \Phi(g_{il})\big)^{t_{il} - m_{il}}\bigg) \\
				& = \sum_{l=1}^{L} \bigg(\log \binom{t_{il}}{m_{il}} + m_{il} \log \Phi(g_{il}) + \big(t_{il} - m_{il} \big) \log \big(1 - \Phi(g_{il})\big)\bigg)
  \end{split}
\end{equation}