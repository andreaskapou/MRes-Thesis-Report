\section{Motivation} \label{motivation-intro-l}
Modern high-throughput genomics platforms generate large amounts of biomedical data from different sources, and these data are used to measure diverse, but often related and complementary, information. Even though these data are becoming widely available, integrated computational analysis is still a challenging task and crucial in order to interpret and uncover biological regulatory mechanisms \cite{Park2009}. 

The main aim of this project, is to integrate the heterogeneous types of high-throughput biological data using an unsupervised integrative data modelling approach. The standard approaches in integrative clustering either perform a separate clustering followed by a post hoc integration or incorporate all data types simultaneously and generate a single 'joint' clustering. Separate clusterings may not be able to capture inter-source associations of the data and on the other hand 'joint' clustering ignores the heterogeneity of the data sources and it also cannot capture source-specific features. Thus, flexible clustering methods need to be developed that can simultaneously integrate information from the different data and can also capture the underlying structural similarities across the data sources.

This project is mainly motivated by the work of \cite{Lock2013}, who proposed an integrative statistical model, named \emph{Bayesian Consensus Clustering} (BCC). BCC simultaneously models the dependence and the heterogeneity of the data sources, by assuming that there is a separate clustering of the objects for each data source, but these source-specific clusterings adhere loosely to an overall consensus clustering. 

The BCC model was implemented under the assumption that all the different data sources were following a Gaussian distribution, which can be a valid assumption when the data were created from hybridization experiments. But, with the advent of high-throughput sequencing techniques, the generated data follow different probability models, since most of the experiments return \emph{count data}. Thus, the main goal of this project is to extend the BCC model so it can perform integrative clustering with various probability models, such as Poisson and Binomial distributions. We will use the rich Encyclopedia of DNA Elements (ENCODE) \cite{Dunham2012} datasets, and sources like gene expression, DNA methylation and histone modification will be mainly used.