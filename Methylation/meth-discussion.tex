\section{Discussion} \label{meth-discussion-sect}
The main contribution of this chapter is to propose a novel approach for modelling and clustering DNA methylation profiles around genomic regions of interest, such as promoters. This approach is important for the biology community, since recent studies have suggested a potentially functional role for methylation profile patterns.

We model methylation data generated from NGS technology using a parametric approach by selecting an explicit set of latent basis functions. Since each CpG site is modelled with a Binomial distribution and a \emph{probit} transformation of the unconstrained latent functions is used to map them to the [0,1] interval, the observation model is termed as the \emph{Binomial distributed Probit regression} model. Finally, a mixture model approach is chosen to cluster together methylation profiles, and EM algorithm is applied to estimate model parameters. 

Preliminary results with two real datasets, produced by the ENCODE project, confirm the association between DNA methylation profiles and gene expression. On both datasets, similar methylation patterns have similar effects on the expression levels of the corresponding genes. For example, U-shaped methylation profiles around the TSS, in general result in higher transcriptional activity. 

Even though these preliminary results are promising, a more comprehensive analysis of the method needs to be exploited, which was not feasible due to time constraint of this project. For instance, it would be more meaningful, from a biological viewpoint, to evaluate the performance of the proposed method on experiments that measure DNA methylation on same cells but on different biological conditions (\eg normal \emph{versus} cancer tissues), and then investigate the relationship between difference in methylation profiles and gene expression changes. Furthermore, a \emph{Gene Ontology} (GO) \citep{Ashburner2000} analysis would be useful, since it would enable functional interpretation of genes that were assigned to different clusters. With this analysis, different methylation profiles would be associated with different biological processes of direct clinical relevance, \eg oncogene activation or genomic imprinting \citep{Li1993}. 

Finally, it should be mentioned that the proposed method has inherent limitations similar to any fixed window-based approach. The distribution of CpG sites in the promoter regions is not uniform, hence, many sub-regions inside those regions may be empty of CpGs; making the inference task of fitting the latent functions to the data even harder, since there are no data points from which the functions could \emph{learn} their final shape. To ameliorate this issue, regions that contain less than $n$ CpG sites should be discarded. Another potential issue, is location \emph{shifting} of similar methylation profiles relative to the TSS. For example, two different promoters could have a similar U-shape profile, but one of them would not be around TSS, but shifted $n$ base pairs relative to TSS. This would result in assigning those similar methylation profiles to different clusters. 