\section{Discussion} \label{integr-discussion-sect}
In this chapter we considered the problem of performing integrative clustering on multisource data, and more specifically on data that are produced from Next Generation Sequencing technology.

BCC model, proposed by \citet{Lock2013}, is a flexible clustering method that simultaneously integrates information from different data and also captures the underlying structural similarities across data sources. As it was shown, BCC can be thought as a flexible bridge between the extremes of separate and joint clusterings, since it introduces a parameter for controlling the adherence of each data source to the overall clustering. Also, by performing an integrative statistical modelling of both the source-specific and overall clusterings, the method can model uncertainty in all parameters simultaneously and also has the ability to borrow information across data sources.

The main contribution of this project was to extend the BCC model to be applicable to NGS genomics data. A \emph{Binomial} observation model was introduced for clustering DNA methylation data generated from RRBS experiments, and a \emph{Poisson} observation model for clustering digital gene expression data generated from RNA-Seq experiments. The analysis that we performed on the synthetic data, confirm the statistical power of the method on estimating the model parameters with high accuracy. We performed approximate inference using Gibbs sampling and in most cases the algorithm was mixing and was rapidly reaching the target distribution. Finally, the label switching problem was ameliorated by applied Stephen's relabelling algorithm \citep{Stephens2000}.

The extended BCC model was also applied on real datasets produces by the ENCODE project. Preliminary results confirmed the relationship between DNA methylation at promoter regions and gene expression. The relationship and dependence between the data sources justifies the use of the BCC model for performing integrative clustering on their common objects; although it should be noted that when this dependence is not expected or an overall clustering is not sought, simpler models may be more appropriate. 

Finally, similarly to \emph{Chapter \ref{model-meth-chapter}} a more comprehensive analysis of the model needs to be exploited by evaluating the performance of method on more biologically meaningful experiments. Also, a Gene Ontology enrichments analysis would be useful to enable functional interpretation of genes that are assigned to the source-specific or overall clusters.