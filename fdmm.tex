\section{Finite Dirichlet Mixture Models} \label{fdmm-s}

\subsection{Bayesian Statistics}
The use of Maximum Likelihood for inferring the values of the parameters $\Theta$ belongs in the \emph{frequentist} interpretation of probability. In this approach, the parameters $\Theta$ are considered to be fixed, the values are inferred by an estimator, \eg MLE, and we get estimate error bars by considering a distribution of datasets $\mathbf{X}$ \cite[Ch. 1]{Bishop2006}. 

On the other hand, in Bayesian statistics probabilities provide a quantification of uncertainty. In this setting, the dataset $\mathbf{X}$ is observed and hence is fixed, and we express our uncertainty in the model parameters, by considering the parameters themselves as random variables. Our initial beliefs about $\Theta$, before observing the data, are represented by a \emph{prior} probability distribution $p(\Theta)$. The effect of observing the data $\mathbf{X}$ is given through the \emph{likelihood} function $p(\mathbf{X}|\Theta)$, which expresses how probable are the observed data given the parameters. Hence, to update our beliefs about the value of $\Theta$, after observing the data $\mathbf{X}$, we use the machinery of probability theory, and more specifically Bayes' theorem, which is:

\begin{equation}
  \begin{aligned}
	p(\Theta | \mathbf{X}) & = \frac{p(\mathbf{X}|\Theta) p(\Theta)}{p(\mathbf{X})} \\
	& = \frac{p(\mathbf{X}|\Theta) p(\Theta)}{\int p(\mathbf{X}|\Theta) p(\Theta) d\Theta}
  \end{aligned}
\end{equation}
and can be interpreted as follows:
\begin{equation}
	\text{posterior} = \frac{likelihood \times prior}{evidence}
\end{equation}

\subsubsection*{ML Estimation of $\Theta$}
Under the Maximum Likelihood approach we seek the value of $\Theta$ that maximizes the likelihood that is:
\begin{equation} \label{MLE-f-bayes}
	\hat{\Theta} =  \underset{\Theta}{\operatorname{argmax}} \; p(\mathbf{X}|\Theta)
\end{equation}

\subsubsection*{MAP Estimation of $\Theta$}
Under the Maximum a Posteriori approach we seek the value of $\Theta$ that maximizes the posterior $p(\Theta | \mathbf{X})$, that is:
\begin{equation} \label{MAP-f-bayes}
  \begin{aligned}
	\hat{\Theta} & =  \underset{\Theta}{\operatorname{argmax}} \; p(\Theta | \mathbf{X}) \\
	& \propto \underset{\Theta}{\operatorname{argmax}} \; p(\mathbf{X}|\Theta) p(\Theta)
  \end{aligned}
\end{equation}
where the evidence $p(\mathbf{X})$ can be ignored since it does not depend on $\Theta$. Thus, MAP estimation, in contrast to ML, incorporates in the model our prior beliefs regarding the values of the parameters.

\subsubsection*{Bayesian Estimation of $\Theta$}
Under the Bayesian approach we compute the posterior distribution over the parameters, that is:
\begin{equation} \label{posterio-f-bayes}
  \begin{aligned}
	p(\Theta | \mathbf{X}) = \frac{p(\mathbf{X}|\Theta) p(\Theta)}{p(\mathbf{X})} 
  \end{aligned}
\end{equation}

Thus, in the Bayesian paradigm instead of estimating a fixed value for the parameters, we also capture the uncertainty of the estimation by computing the posterior distribution of the parameters. On the other hand, inference in frequentist statistics, \eg ML or MAP, involves in finding an optimum point estimate of the parameters, but they do not account for any uncertainty in the estimated value of the parameters. 

Even though Bayesian framework is appealing, it is limited by practical difficulties since we need to marginalize over the whole parameter space. And in most cases marginalization is computationally difficult since the integral might be intractable and difficult to approximate. To be able to compute Bayesian integrals analytically, \emph{conjugate priors} should be used if possible. For a given functional form of the likelihood $p(\mathbf{X}|\Theta)$, the prior $p(\Theta)$ is said to be conjugate for that likelihood, if the posterior $p(\Theta|\mathbf{X})$ has the same functional form as the prior.

\begin{minipage}{0.6\textwidth}%
  \hfill
  \begin{center}
	% model_pca.tex
%
% Copyright (C) 2012 Jaakko Luttinen
%
% This file may be distributed and/or modified
%
% 1. under the LaTeX Project Public License and/or
% 2. under the GNU General Public License.
%
% See the files LICENSE_LPPL and LICENSE_GPL for more details.

% PCA model

%\beginpgfgraphicnamed{model-pca}
\begin{tikzpicture}

  % Define nodes
  \node[obs]                    (X) {$x_{i}$};
  \node[latent, above=of X] 	(Z) {$z_{i}$};
  \node[latent, above=of Z]  	(p) {$\pi$};
  \node[latent, right=1cm of Z] (t) {$\theta_{k}$};
  \node[latent, above=of t] 	(G) {$G_{0}$};

  % Connect the nodes
  \edge {Z,t} {X} ; %
  \edge {p} {Z} ; %
  \edge {G} {t} ; %

  % Plates
  \plate {} {(t)} {$K$};
  \plate {} {(Z)(X)} {$N$};

\end{tikzpicture}
%\endpgfgraphicnamed

%%% Local Variables: 
%%% mode: tex-pdf
%%% TeX-master: "example"
%%% End: 

	\emph{Graphical Model of FDMM.}
  \end{center}
\end{minipage}
%\hfill
\begin{minipage}{0.1\textwidth}%\raggedright
  \begin{equation*}
  	\begin{aligned}
  		\mathbf{\pi} \; & \sim \; Dir(\mathbf{\delta}) \\
  		z_{i}|\mathbf{\pi} \; & \sim \; Cat(\mathbf{\pi}) \\
  		\theta_{k} \; & \sim \mathcal{G}_{0}; \mathcal{H} \\
  		x_{i}|z_{i}=k,\theta_{k} \; & \sim \; p(\cdot | \theta_{k})  
  	\end{aligned} 
  \end{equation*} 
\end{minipage}

\vspace*{5mm}
The full joint distribution of the FDMM model by looking at the graphical representation factorizes as follows:
\begin{equation}%\scriptstyle
	p(\mathbf{X},\mathbf{Z},\mathbf{\pi},\Theta;\delta,\mathcal{H}) = p(\mathbf{X}|\mathbf{Z},\Theta) p(\mathbf{Z}|\pi) p(\pi|\delta) p(\Theta |\mathcal{G}_{0}; \mathcal{H})
\end{equation}

where $\mathcal{H}$ is the set of all the hyper-parameters of the $\mathcal{G}_{0}$ prior distribution related to the parameters $\Theta$, and $\mathbf{\delta}$ is a K-dimensional vector with the hyper-parameters of the \emph{Dirichlet} prior. 

To do inference in the Bayesian framework, the posterior distribution of the parameters needs to be computed. By applying the Bayes Rule and conditioning on the observed data $\mathbf{X}$, the posterior distribution is simply proportional to the full joint. Thus:
 
\begin{equation}%\scriptstyle
  \begin{aligned}
	p(\mathbf{Z},\mathbf{\pi},\Theta|\mathbf{X} ;\delta,\mathcal{H}) & = \frac{p(\mathbf{X}|\mathbf{Z},\mathbf{\pi},\Theta) p(\mathbf{Z},\mathbf{\pi},\Theta ;\delta,\mathcal{H})}{p(\mathbf{X})} \\
	   & \propto p(\mathbf{X}|\mathbf{Z},\mathbf{\pi},\Theta) p(\mathbf{Z},\mathbf{\pi},\Theta ;\delta,\mathcal{H}) \\
	   & = p(\mathbf{X}|\mathbf{Z},\Theta) p(\mathbf{Z}|\pi) p(\pi|\delta) p(\Theta |\mathcal{G}_{0}; \mathcal{H}) \\
	   & = \bigg(\prod\limits_{i=1}^{N} p(x_{i}|z_{i},\theta_{z_{i}}) p(z_{i}|\pi)\bigg) p(\pi|\delta) \bigg(\prod\limits_{k=1}^{K} p(\theta_{k} |\mathcal{G}_{0}; \mathcal{H})\bigg)
  \end{aligned}
\end{equation}
where we use the fact that $\mathbf{X}$ is conditionally independent of $\pi$ given $\mathbf{Z}$, \ie $\mathbf{X} \bigCI \pi \;| \; \mathbf{Z} $. 

This posterior is intractable to compute, thus we need to resort to approximation schemes, and these fall broadly into two main classes. Deterministic approximation schemes, such as \emph{Variational Bayes}, are based on analytical approximations the posterior distribution, by assuming that it factorizes in a particular way or that the posterior will have a specific form, such as Gaussian \cite[Ch. 10]{Bishop2006}.  
