\section{Bayesian Consensus Clustering} \label{integr-bcc-sect}
Motivated by the work of \citet{Lock2013}, this project is concerned in extending BCC to be applicable to NGS genomics data. Initial method was mainly intended for modelling \emph{continuous data}, and more specifically the observation model was assumed to follow a Gaussian distribution. Thus, the approach was demonstrated mainly on TCGA datasets, which mainly were generated from \emph{microarray hybridization} experiments. But, with the advent of high-throughput sequencing methods, the data follow different probability models, since most experiments return \emph{count data}, as it was explained in \emph{Chapter \ref{dataset-chapter}}. Thus, we need to extend BCC, so it can model data that follow different probability distributions, such as \emph{Binomial} and \emph{Poisson}.

For the explanation of the BCC model, notation will be the same as the one used in \citet{Lock2013}. 

To perform integrative clustering for M different data sources $\mathbb{X}_{1},..., \mathbb{X}_{M}$, the Finite Dirichlet Mixture Model (FDMM) (see \emph{Section \ref{back-fdmm-s}}) is extended, which leads us to the Bayesian Consensus Clustering model. BCC assumes that there are N common objects for each data source, where $X_{mn}$ denotes the data for object $n$ from source $m$. The data sources $\mathbb{X}_{m}$ can have any disparate structure, and each of them requires an arbitrary observation model $f_{m}(X_{mn}|\theta_{m})$, where $\theta_{m}$ denotes the model parameters for data source $m$.

Let $\mathbb{L}_{m} = (L_{m1},...,L_{mN})$ denote the source specific clusterings and $\mathbb{C} = (C_{1},...,C_{N})$ denote the overall clustering of the data. The model assumes that both $\mathbb{C}$ and $\mathbb{L}_{m}$ will have the same number of possible clusters K, thus $L_{mn} \in \lbrace 1,...,K \rbrace$ and $C_{n} \in \lbrace 1,...,K \rbrace$, will denote the source-specific and the overall mixture components for observation $X_{mn}$, respectively.

The source specific clusterings and the overall clustering are related to each other, and the strength of the relation is given by a dependence function $v(L_{mn}, C_{n}, \alpha_{m})$, where $\alpha_{m} \in [\frac{1}{K}, 1]$ is the adherence parameter and controls the adherence of data source $\mathbb{X}_m$ to the overall clustering $\mathbb{C}$. More simply, it explains how much does the source specific clustering for source $m$ agrees with the overall clustering $\mathbb{C}$. So if $\alpha_{m} = 1$, then $\mathbb{L}_{m} = \mathbb{C}$. 

The dependence function $v$ has the form:

\begin{equation}
	v(L_{mn}, C_{n}, \alpha_{m}) = \left\{
	\begin{array}{l l}
		\alpha_{m},\quad \quad \quad if\quad C_{n} = L_{mn}\\
		\frac{1-\alpha_{m}}{K-1},\quad \quad otherwise
	\end{array}\right.
\end{equation}

The graphical representation for the BCC model is given in \emph{Figure \ref{bcc-pic}}, where again $G_{0}$ is the prior for the parameters $\theta_{mk}$.

\begin{minipage}{0.5\textwidth}%
  \hfill
  \begin{center}
	% model_pca.tex
%
% Copyright (C) 2012 Jaakko Luttinen
%
% This file may be distributed and/or modified
%
% 1. under the LaTeX Project Public License and/or
% 2. under the GNU General Public License.
%
% See the files LICENSE_LPPL and LICENSE_GPL for more details.

% PCA model

%\beginpgfgraphicnamed{model-pca}
\begin{tikzpicture}

  % Define nodes
  \node[obs]                              (X) {$X_{mn}$};
  \node[latent, above=of X] 		(L) {$L_{mn}$};
  \node[latent, above=of L] 		(a) {$\alpha$};
  \node[latent, right=1.4cm of L]  (C) {$C_{n}$};
  \node[latent, above=of C] 		(p) {$\pi$};
  \node[latent, right=1cm of C]  (t) {$\theta_{mk}$};
  \node[latent, above=of t] 		(G) {$\mathcal{G}_{0}$};

  % Connect the nodes
  \edge {L,t} {X} ; %
  \edge {a,C} {L} ; %
  \edge {p} {C} ; %
  \edge {G} {t} ; %

  % Plates
  \plate {} {(t)} {$K$};
  \plate {yx} {(L)(X)} {$M$};
  \plate {} {(C)(X)(yx.north west)(yx.south west)} {$N$};

\end{tikzpicture}
%\endpgfgraphicnamed

%%% Local Variables: 
%%% mode: tex-pdf
%%% TeX-master: "example"
%%% End: 

	\emph{Graphical Model of BCC.}
  \end{center}
\end{minipage}
%\hfill
\begin{minipage}{0.4\textwidth}%\raggedright
  \begin{equation*}
  	\begin{aligned}
  		\mathbf{\pi} \; & \sim \; \mathcal{D}ir(\delta) \\
  		\alpha \; & \sim \; \mathcal{TB}(\mathit{a}, \beta, \frac{1}{K}) \\
  		\theta_{mk} \; & \sim \mathcal{G}_{0}; \mathcal{H} \\
  		C_{n}|\mathbf{\pi} \; & \sim \; \mathcal{C}at(\mathbf{\pi}) \\
  		L_{mn} | \alpha_{m}, C_{n} \; & \sim \; v(L_{mn}, C_{n}, \alpha_{m}) \\
  		X_{mn}|L_{mn}=k,\theta_{mk} \; & \sim \; f_{m}(\cdot | \theta_{mk}) 
  	\end{aligned} 
  \end{equation*} 
\end{minipage}
\vspace*{7mm}

The full joint distribution of the model by looking at the graphical representation in \emph{Fig \ref{bcc-pic}} is given by:

\begin{equation}\scriptstyle
	P(\mathbb{X}, \mathbb{L}, \mathbb{C}, \Pi , \Theta , \alpha ; \mathcal{H}) = P(\mathbb{X}|\mathbb{L},\Theta) P(\mathbb{L}|\mathbb{C},\alpha) P(\mathbb{C}|\Pi) P(\Pi) P(\alpha) P(\Theta ; \mathcal{H})
\end{equation}

where $\mathcal{H}$ is the set of all the hyper-parameters, related to the parameters $\Theta$. 

To do inference in the Bayesian framework, the posterior distribution of the parameters needs to be computed. By applying the Bayes Rule and conditioning on the observed data $\mathbb{X}$, the posterior distribution is simply proportional to the full joint. Thus:
 
\begin{equation}\scriptstyle
	\begin{aligned}
	& & \scriptstyle P(\mathbb{L},\mathbb{C},\Pi ,\Theta , \alpha | \mathbb{X}; \mathcal{H}) & \scriptstyle \propto P(\mathbb{X}|\mathbb{L}, \mathbb{C},\Pi ,\Theta , \alpha) P(\mathbb{L}, \mathbb{C}, \Pi , \Theta , \alpha; \mathcal{H}) \\
	& & & \scriptstyle = P(\mathbb{X}|\mathbb{L},\Theta) P(\mathbb{L}|\mathbb{C},\alpha) P(\mathbb{C}|\Pi) P(\Pi) P(\alpha) P(\Theta ; \mathcal{H})
	\end{aligned}
\end{equation}

This posterior is intractable to compute, thus we need to resort to approximation schemes and these fall broadly into two classes. Deterministic approximation schemes such as \emph{variational inference} could be used, which are based on analytically approximating the posterior distribution, by assuming that it factorizes in a particular way, or that the posterior will have a specific parametric form, such as a Gaussian, which then we try to approximate.

The other approximation scheme, and the one that is used in this project, is a stochastic approximation of the posterior using Markov Chain Monte Carlo (MCMC) methods. Specifically, \emph{Gibbs sampling} algorithm will be used, which is one of the most popular MCMC algorithms. 

Deriving Gibbs sampling for this model, requires deriving an expression for the \emph{full conditional distribution} of every latent variable conditioned on all the others. Thus, from the graphical model shown in \emph{Figure \ref{bcc-pic}}, we can infer the dependencies of the latent variables by looking at the \emph{Markov blanket} of each variable, which are its neighbours in the graph. 

For mathematical convenience, \emph{conjugate priors} should be used which would allow us to perform analytically some marginalization steps that are necessary in deriving Gibbs sampler. Thus:

\begin{itemize}
	\item $\Pi \sim Dir(\mathbf{\delta_{0}})$. For the mixing proportions a natural choice is a Dirichlet prior distribution, and the hyper-parameter vector $\mathbf{\delta_{0}}$ is set to $(1,...,1)$ so the prior is uniformly distributed in the $K-1$ simplex.
	\item $\alpha_{m} \sim TBeta(\mathit{a_{m}}, \beta_{m}, \frac{1}{K})$. Which means that $a_{m}$ follows a  $Beta$ distribution with parameters $\mathit{a_{m}}, \beta_{m}$ truncated below by $\frac{1}{K}$.
	\item $\theta_{mk} \sim G_{0}$. Where $G_{0}$ is a conjugate prior distribution so that sampling from the conditional posterior $p_{m}(\theta_{mk}|\mathbb{X}_{m}, \mathbb{L}_{m})$ is feasible. For example, if the probability model $f_{m}(X_{mn}|\theta_{m}) \sim Poisson(\lambda_{m})$, then a conjugate prior distribution would be a $Gamma(\lambda_{m}|\mathit{a_{m}}, \beta_{m})$.
\end{itemize}

Having chosen the appropriate prior distributions, the Gibbs sampler would proceed iteratively by sampling from the following full conditional distributions:

\begin{enumerate}
	\item $\scriptstyle \mathbb{L}_{m} | \mathbb{X}_{m}, \Theta_{m}, \mathbb{C}, \alpha_{m} \sim v(L_{mn}, C_{n}, \alpha_{m}) f_{m}(X_{mn}|\theta_{mk}) \quad \text{for} \quad n=1,...,N $
	\item $\scriptstyle \Theta_{m}|\mathbb{L}_{m}, \mathbb{X}_{m} \sim p_{m}(\theta_{mk}|\mathbb{X}_{m}, \mathbb{L}_{m}) \quad \text{for} \quad k=1,...,K $
	\item $\scriptstyle \alpha_{m}|\mathbb{L}_{m}, \mathbb{C} \sim TBeta \big(\mathit{a_{m}}+\sum_{n=1}^{N}\mathbbm{1}(L_{mn}=k,C_{n}=k), \beta_{m}-N-\sum_{n=1}^{N}\mathbbm{1}(L_{mn}=k,C_{n}=k), \frac{1}{K}\big)$
	\item $\scriptstyle \mathbb{C}|\mathbb{L}_{m}, \Pi, \alpha \sim \pi_{k}\prod_{m=1}^{M}v(L_{mn, C_{n}, \alpha_{m}}) \quad \text{for} \quad n=1,...,N$
	\item $\scriptstyle \Pi|\mathbb{C} \sim Dirichlet(\{ \delta_{k} + \sum_{n=1}^{N}\mathbbm{1}(C_{n}=k)\}_{k=1}^{K })$
\end{enumerate}