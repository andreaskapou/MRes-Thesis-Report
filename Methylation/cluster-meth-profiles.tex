\section{Clustering DNA methylation profiles} \label{cluster-meth-s}

\subsection{EM for clustering methylation profiles}
In this section, we will derive the \emph{E} and \emph{M} steps for clustering methylation profiles using the model described in \emph{Section \ref{model-meth-profiles-s}}.

But before, we have to explicitly introduce the parameters $\theta_{k}$ of the model, excluding the mixing proportions $\pi_{k}$ which are present for any mixture model. For example, in a Gaussian Mixture Model (GMM) the model parameters $\theta_{k}$, are the mean $\mu_{k}$ and the variance $\sigma_{k}^{2}$ of the Gaussian components. In our model, the Binomial distributed Probit regression function given in \emph{Eq. \ref{likel-binom-prob-f}}, the parameters are the coefficients of the $n^{th}$ degree polynomial function $\mathbf{g}_{i}$. Thus, during the EM algorithm we infer these sets of parameters for each cluster $k$. For example, for a $2^{nd}$ degree polynomial, we can rewrite \emph{Eq. \ref{likel-binom-prob-f}} by explicitly showing the model parameters as follows:
\begin{equation} \label{likel-binom-prob-example1-f}
  \begin{split}
	p(\mathbf{y}_{i}|\mathbf{f}_{i}, \theta) & = \prod_{l=1}^{L} p(y_{il}|f_{il}, \theta) \\
							 & = \prod_{l=1}^{L} Binom \big(t_{il}, \Phi(g_{il}; \theta)\big) \\
							 & = \prod_{l=1}^{L} \binom{t_{il}}{m_{il}} \Phi(\alpha x_{il}^{2} + \beta x_{il} + c)^{m_{il}} (1 - \Phi(\alpha x_{il}^{2} + \beta x_{il} + c)\big)^{t_{il} - m_{il}}
  \end{split}
\end{equation}

where $\theta = (\alpha, \beta, c)$.

\subsubsection{E-Step}
In the E-step we compute the responsibility that component k takes on explaining observations $\mathbf{y_{i}}$, and by substituting the proposed likelihood function to \emph{Eq.}, we have:
\begin{equation} \label{responsibilities-binom-prob-model-f}
  \begin{split}
	\gamma(z_{ik}) & \triangleq p(z_{i}=k|\mathbf{y_{i}},\theta_{k}^{t-1}) \\
				   & = \frac{\pi_{k}p(\mathbf{y}_{i}|\mathbf{f}_{i},z_{i}=k,\theta_{k}^{t-1})}{\sum\limits_{j=1}^{K} \pi_{j}p(\mathbf{y}_{i}|\mathbf{f}_{i},z_{i}=j,\theta_{j}^{t-1})} \\
				   & = \frac{\pi_{k} \prod\limits_{l=1}^{L} Binom \big(t_{il}, \Phi(g_{il}; \theta_{k}^{t-1})\big)} {\sum\limits_{j=1}^{K} \pi_{j} \prod\limits_{l=1}^{L} Binom \big(t_{il}, \Phi(g_{il}; \theta_{j}^{t-1})\big)}
  \end{split}
\end{equation}

\subsubsection{M-Step}
In the M-step, we optimize \emph{Eq. } with respect to $\pi_{k}$ and $\theta_{k}$. The update for the mixing proportions $\pi_{k}$ on each M-step, is the same as shown in \emph{Eq. \ref{mixing-proportions-binom-prob-est-f}}, that is:
\begin{equation} \label{mixing-proportions-binom-prob-est-f}
		\pi_{k} = \frac{1}{N} \sum_{i} \gamma(z_{ik})
\end{equation}

To update the parameters $\theta_{k}$, we need to optimize $\ell(\theta_{k})$. Substituting our model to \emph{Eq. \ref{parameters-est2-binom-prob-EM-f}}, the quantity that needs to be maximized is:
\begin{equation} \label{parameters-est2-binom-prob-EM-f}
  \begin{split}
	\ell(\theta_{k}) & \triangleq \sum_{i} \sum_{k} \gamma(z_{ik}) \log p(\mathbf{y}_{i}|\mathbf{f}_{i}, \theta_{k}) \\
					 & = \sum_{i} \sum_{k} \gamma(z_{ik}) \log \bigg( \prod_{l} Binom \big(t_{il}, \Phi(g_{il}; \theta_{k})\big) \bigg)\\
					 & = \sum_{i} \sum_{k} \gamma(z_{ik}) \sum_{l} \log \bigg(Binom \big(t_{il}, \Phi(g_{il}; \theta_{k})\big) \bigg)
  \end{split}
\end{equation}

Direct maximization of this quantity w.r.t the parameters $\theta_{k}$ is not feasible, thus Conjugate Gradients method \citep{Hestenes1952} will be used, which is a first order numerical optimization algorithm for solving particular systems of linear equations. Thus, it requires to compute the \emph{gradient}, which for the example of the $2^{nd}$ degree polynomial is:

\begin{equation} \label{gradient-f}
	\nabla\ell = \bigg( \frac{\partial \ell(\theta_{k})}{\partial \alpha}, \frac{\partial \ell(\theta_{k})}{\partial \beta}, \frac{\partial \ell(\theta_{k})}{\partial c}\bigg) 
\end{equation}
where the partial derivatives are given in the equations below (see Appendix ... for the whole derivations):
\begin{equation} \label{derivative-a-f}
	\frac{\partial \ell(\theta_{k})}{\partial \alpha} =  \sum_{i}  \gamma(z_{ik}) \sum_{l} \bigg[ x_{il}^{2} \mathbf{\phi}(g_{il};\theta_{k})\bigg(\frac{m_{il} - t_{il}\Phi(g_{il};\theta_{k})}{\Phi(g_{il};\theta_{k})\big(1-\Phi(g_{il};\theta_{k})\big)} \bigg) \bigg]
\end{equation}

\begin{equation} \label{derivative-b-f}
	\frac{\partial \ell(\theta_{k})}{\partial \beta} =  \sum_{i}  \gamma(z_{ik}) \sum_{l} \bigg[ x_{il} \mathbf{\phi}(g_{il};\theta_{k})\bigg(\frac{m_{il} - t_{il}\Phi(g_{il};\theta_{k})}{\Phi(g_{il};\theta_{k})\big(1-\Phi(g_{il};\theta_{k})\big)} \bigg) \bigg]
\end{equation}

\begin{equation} \label{derivative-c-f}
	\frac{\partial \ell(\theta_{k})}{\partial c} =  \sum_{i}  \gamma(z_{ik}) \sum_{l} \bigg[ \mathbf{\phi}(g_{il};\theta_{k})\bigg(\frac{m_{il} - t_{il}\Phi(g_{il};\theta_{k})}{\Phi(g_{il};\theta_{k})\big(1-\Phi(g_{il};\theta_{k})\big)} \bigg) \bigg]
\end{equation}
where $\mathbf{\phi}$ is the \emph{probability density function (pdf)} for the standard normal distribution $\mathcal{N}(0,1)$.