\section{Motivation} \label{motivation-intro-l}
Interpreting the genome sequence of different species is one of the major challenges in genetic and biological research. The advent of high-throughput sequencing platforms for studying and extracting genetic information from biological systems, has triggered ground-breaking discoveries and revolutionized our understanding for the \emph{genome} and \emph{epigenome} of many species. 

In particular, \emph{RNA-Seq} experiments \citep{Marioni2008} have rivaled microarrays, and now are extensively used for trancriptome profiling. \emph{Chip-Seq} \citep{Park2009} is used to quantitatively measure and analyse protein interactions with DNA, \ie histone modifications and transcription factors. \emph{RRBS} \citep{Meissner2005} uses bisulphite treatment of DNA and allows estimation of methylation level at a single-nucleotide resolution. These are only some examples of different platforms and techniques that are used to measure diverse biological components. Despite the widespread take up of the high-throughput sequencing technology, computational data analysis is still a challenging task and crucial in order to interpret and uncover biological regulatory mechanisms \citep{Park2009}.

A \emph{model} is an abstraction of a real system and is designed to capture regularities in the observed data and subsequently make accurate predictions. Real systems, and especially biological systems, are far too complex for the modeller to design an accurate representation of the process that generated the data, and in most cases they may contain noisy and incomplete observations. Hence, simpler models are often sought that can approximate the true processes that generated the data. The \emph{machine learning} approach to constructing flexible models is to introduce a set of \emph{parameters} that specify the model and then seek a setting for those parameters that explains the data best. The idea is that if we can explain the data well by this setting of parameters, then we can also be confident when making predictions for future observations. The procedure of finding the best setting of the model parameters for the observed data is called \emph{learning} the model. A \emph{mixture model} is a widely used probabilistic model for performing descriptive analysis. It can be applied to different tasks, such as density estimation or even for predictions, but its main application is for performing \emph{clustering}, that is, identifying similar objects and group them together in clusters. 

Our main aim in conducting this project is to perform mixture modelling on high-throughput genomics data on two different applications. Initially, we propose a novel approach for modelling and clustering DNA methylation profiles around genomic regions of interest, such as promoter regions. Recent studies have suggested that the shape of methylation profiles plays an important role in predicting gene expression, leading to a potentially functional role for methylation profiles. 



%The advent of high-throughput sequencing platforms for studying the regulation of gene expression has revolutionized our understanding for the importance of the epigenomic 'marks', such as DNA methylation and histone modifications, in cellular processes.  