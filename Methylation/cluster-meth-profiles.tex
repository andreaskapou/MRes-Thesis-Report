\section{Clustering DNA Methylation Profiles} \label{cluster-meth-s}
After defining a model for the DNA methylation profiles, a statistical method needs to be proposed in order to perform model-based clustering of methylation profiles. 

A mixture model approach \citep{McLachlan1988} is chosen to cluster methylation profiles, and the model is fitted to the data using maximum likelihood. The \emph{EM algorithm} \citep{Dempster1977} is chosen to estimate the mixture model parameters. Below, we derive the \emph{E} and \emph{M} steps for clustering methylation profiles using the \emph{Binomial distributed Probit regression} function as our observation model. 

But before, we have to explicitly introduce the parameters $\theta_{k}$ of the observation model, excluding the mixing proportions $\pi_{k}$ which are present for any mixture model. For example, in a Gaussian Mixture Model (GMM) the model parameters $\theta_{k}$ are the mean $\mu_{k}$ and the variance $\sigma_{k}^{2}$ of the Gaussian components. In our model, the Binomial distributed Probit regression function, the parameters are the coefficients of the $n^{th}$ degree polynomial function $\mathbf{g}_{i}$, and the EM algorithm infers these sets of parameters for each cluster $k$. 

For example, for a $2^{nd}$ degree polynomial, we can rewrite \emph{Eq. \ref{likel-binom-prob-f-meth}} by explicitly showing the model parameters as follows:
\begin{equation} \label{likel-binom-prob-example1-f-meth}
  \begin{split}
	p(\mathbf{y}_{i}|\mathbf{f}_{i}, \theta) & = \prod_{l=1}^{L} p(y_{il}|f_{il}, \theta) \\
							 & = \prod_{l=1}^{L} Binom \big(t_{il}, \Phi(g_{il}; \theta)\big) \\
							 & = \prod_{l=1}^{L} \binom{t_{il}}{m_{il}} \Phi(\alpha x_{il}^{2} + \beta x_{il} + c)^{m_{il}} (1 - \Phi(\alpha x_{il}^{2} + \beta x_{il} + c)\big)^{t_{il} - m_{il}}
  \end{split}
\end{equation}
where $\theta = (\alpha, \beta, c)$.

\subsection*{E-step}
In the E-step, we compute the responsibility that component k takes on explaining observations $\mathbf{y}_{i}$, and by substituting the proposed likelihood function to \emph{Eq. \ref{responsibilities-f-mm}}, we have:
\begin{equation} \label{responsibilities-binom-prob-model-f-meth}
  \begin{split}
	\gamma(z_{ik}) & \triangleq p(z_{i}=k|\mathbf{y}_{i},\theta_{k}) \\
				   & = \frac{\pi_{k}p(\mathbf{y}_{i}|\mathbf{f}_{i},z_{i}=k,\theta_{k})}{\sum\limits_{j=1}^{K} \pi_{j}p(\mathbf{y}_{i}|\mathbf{f}_{i},z_{i}=j,\theta_{j})} \\
				   & = \frac{\pi_{k} \prod\limits_{l=1}^{L} Binom \big(t_{il}, \Phi(g_{il}; \theta_{k})\big)} {\sum\limits_{j=1}^{K} \pi_{j} \prod\limits_{l=1}^{L} Binom \big(t_{il}, \Phi(g_{il}; \theta_{j})\big)}
  \end{split}
\end{equation}
Thus, the E-step essentially remains the same, with the only difference that at each iteration we need to evaluate pointwise a different observation model. 

\subsection*{M-step}
In the M-step, we need to optimize \emph{Eq. \ref{log-lik-expected-derivation-f-mm}} with respect to $\pi_{k}$ and $\theta_{k}$. The update for the mixing proportions $\pi_{k}$ at each M-step remains the same as the simple mixture model, that is:
\begin{equation} \label{mixing-proportions-binom-prob-est-f-meth}
		\pi_{k} = \frac{1}{N} \sum_{i} \gamma(z_{ik})
\end{equation}

To update the parameters $\theta_{k}$, we need to optimize the following quantity:
\begin{equation} \label{parameters-est2-binom-prob-EM-f-meth}
  \begin{split}
	\ell(\theta_{k}) & \triangleq \sum_{i} \sum_{k} \gamma(z_{ik}) \log p(\mathbf{y}_{i}|\mathbf{f}_{i}, \theta_{k}) \\
					 & = \sum_{i} \sum_{k} \gamma(z_{ik}) \log \bigg( \prod_{l} Binom \big(t_{il}, \Phi(g_{il}; \theta_{k})\big) \bigg)\\
					 & = \sum_{i} \sum_{k} \gamma(z_{ik}) \sum_{l} \log \bigg(Binom \big(t_{il}, \Phi(g_{il}; \theta_{k})\big) \bigg)
  \end{split}
\end{equation}

Direct maximization of this quantity w.r.t the parameters $\theta_{k}$ is not feasible, thus Conjugate Gradients method \citep{Hestenes1952} will be used, which is a first order numerical optimization algorithm for solving particular systems of linear equations. Hence, it requires to compute the \emph{gradient}, which for the example of the $2^{nd}$ degree polynomial is:

\begin{equation} \label{gradient-f}
	\nabla\ell = \bigg( \frac{\partial \ell(\theta_{k})}{\partial \alpha}, \frac{\partial \ell(\theta_{k})}{\partial \beta}, \frac{\partial \ell(\theta_{k})}{\partial c}\bigg) 
\end{equation}
where the partial derivatives are given in the equations below (see \emph{Appendix \ref{derivatives-m-step-chapter}} for the whole derivations):
\begin{equation} \label{derivative-a-f}
	\frac{\partial \ell(\theta_{k})}{\partial \alpha} =  \sum_{i}  \gamma(z_{ik}) \sum_{l} \bigg[ x_{il}^{2} \mathbf{\phi}(g_{il};\theta_{k})\bigg(\frac{m_{il} - t_{il}\Phi(g_{il};\theta_{k})}{\Phi(g_{il};\theta_{k})\big(1-\Phi(g_{il};\theta_{k})\big)} \bigg) \bigg]
\end{equation}

\begin{equation} \label{derivative-b-f}
	\frac{\partial \ell(\theta_{k})}{\partial \beta} =  \sum_{i}  \gamma(z_{ik}) \sum_{l} \bigg[ x_{il} \mathbf{\phi}(g_{il};\theta_{k})\bigg(\frac{m_{il} - t_{il}\Phi(g_{il};\theta_{k})}{\Phi(g_{il};\theta_{k})\big(1-\Phi(g_{il};\theta_{k})\big)} \bigg) \bigg]
\end{equation}

\begin{equation} \label{derivative-c-f}
	\frac{\partial \ell(\theta_{k})}{\partial c} =  \sum_{i}  \gamma(z_{ik}) \sum_{l} \bigg[ \mathbf{\phi}(g_{il};\theta_{k})\bigg(\frac{m_{il} - t_{il}\Phi(g_{il};\theta_{k})}{\Phi(g_{il};\theta_{k})\big(1-\Phi(g_{il};\theta_{k})\big)} \bigg) \bigg]
\end{equation}
where $\mathbf{\phi}$ is the \emph{probability density function} for the standard normal distribution $\mathcal{N}(0,1)$.

\subsection{EM initialization}
EM algorithm is sensitive to the choice of the initial parameters. Hence, careful initialization would result in faster convergence to local maximum or even in finding better local maxima. When using simple mixture models, like GMMs, the parameters themselves have physical interpretation, \eg the mean value of the data, thus, \emph{k-means} algorithm can be used directly on the observed data to initialize the parameters.

The model proposed in this work, is more complex since each object is a vector of observations, and these vectors may also be of different length. To provide a sensible initialization for the parameters, the following procedure is taken:
\begin{itemize}
	\item{For each object, we fit a latent function to its observations using \emph{Conjugate Gradients} algorithm to obtain a coefficient vector of the polynomial.}
	\item{After extracting all the coefficient vectors for each object, we run \emph{k-means} on them. The output of \emph{k-means}, will be the initial values for the model parameters.}
\end{itemize}
